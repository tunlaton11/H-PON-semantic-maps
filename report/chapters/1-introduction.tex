\chapter{Introduction}


\section{Background}
With the advancement of technology, automotive technology focuses on not only moving passengers from one place to another but also convenience and safety. This leads to autonomous driving which provides convenience to passengers and road users. There are various components for autonomous driving ranging from environment understanding, route planing, navigating, to self-driving.

For sensing the environment, autonomous driving relies on object detection with the use of camera, distance and position calculation with the use of lidar and ultrasonic sensors. Lidar's cons are its weight and the cost while camera is relatively cheaper and it also has potential to distance and position calculation.


\section{Objective}
To develop the model which is capable of translating images into maps for autonomous driving.


\section{Scope of Work}
The model development focuses on deep learning approach, using a nuScenes dataset, a public dataset for autonomous driving developed by Motional AD Inc, as training data.


\section{Research Implication}
The model would be applied in an autonomous driving car, to reduce the cost of using lidar sensor.

